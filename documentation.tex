\documentclass[11pt]{article}

\usepackage{listings}
\usepackage[draft,pdfcomment]{commenting}

%\usepackage{}

\lstset{
	%language=TeX,
	basicstyle={\ttfamily\small}, % \ttfamily
    keywordstyle={\color{blue!30!black}\bf},
    keywordstyle={[2]\color{red!30!black}},
	xleftmargin=20pt,
    tabsize=4,
    showstringspaces=false,
    columns=flexible,
    breaklines=false,
    fontadjust,
    emphstyle={\color{red!30!black}},
    emphstyle={[2]\color{violet}},
    emphstyle={[3]\rmfamily\it},
    moredelim=[is][emphstyle3]{/}{/},
    moredelim=**[is][emphstyle]{\#}{\#},
    moredelim=**[is][emphstyle2]{@}{@},
    literate={
        {<}{{$\langle$}}1
        {>}{{$\rangle$}}1
        {_0}{{$_0$}}1
        {_1}{{$_1$}}1
        {_n}{{$_n$}}1
        {_k}{{$_k$}}1
    }
}
\lstdefinestyle{noslash}{deletedelim=[is]{/}{/},moredelim=[is][emphstyle3]{!}{!}}
\lstdefinestyle{mp}{basicstyle={\ttfamily\scriptsize}}

\let\cmd=\lstinline
\def\cmdmp{\lstinline[style=mp]}
\lstnewenvironment{latex}[1][]{\lstset{breaklines,#1}}{}
\lstnewenvironment{latexex}[1][]{\lstset{breaklines,style=noslash,#1}}{}

\declareauthor{fz}{Frank}{blue}
\authorcommand{fz}{draftnote}
\declareauthor{jlp}{Jean-Luc}{green!30!black}

\setlength{\marginparwidth}{3.3cm}
\newlength\mparw
\setlength{\mparw}{3.2cm}
\setlength{\parskip}{1.5ex}
\setlength{\parindent}{0ex}

%\usepackage{multicol}

\newcommand{\docwarning}[2]{%
	\marginpar[%
		\leavevmode%
		\vrule width .7pt~%
		\vbox{\hsize=\mparw%
		\raggedleft%
		\scriptsize%
		\it\textbf{#1} #2}%
	]{%
		\leavevmode%
		\vrule width .7pt~%
		\vbox{\hsize=\mparw%
		\raggedright%
		\scriptsize%
		\it\textbf{#1} #2}%
	}%
}

\newcommand{\doccmd}[2][]{%
	\marginpar[%
		\leavevmode%
		\raggedleft%
		\fbox{\cmdmp'#2'}%
		\scriptsize\it\ #1%
	]{%
		\leavevmode%
		\raggedright%
		\fbox{\cmdmp'#2'}%
		\scriptsize\it\ #1%
	}%
}

\begin{document}

\author{Emanuele D'Osualdo}
\title{The \texttt{commenting} package v1.3}
\maketitle
%
%\begin{multicols}{2}
%\tableofcontents
%\end{multicols}

This package attempts to provide a simple and flexible set of macros to typeset
(and hide) comments and remarks which are supposed to be read only by the
authors of a multi-author document, during the process of writing and reviewing
it. It provides macros for printing comments inline or in the margin, giving
visual hints of their location and author. All the comments are typeset only
when in draft mode. Pdf's native comments are also somehow optionally supported
through the \cmd'pdfcomment' package.


Package options are provided to globally change aspects of the appearance of the
comments and the use of more sophisticated features (color, pdfcomment, margin
notes). This is especially useful, for instance, when switching from a normal
layout to a two columns one: setting the \cmd'nompar' option would suppress any
margin note and would include all the comments inline.

\section{Setup}

The \texttt{commenting} package comes in a self contained single
\verb'.sty' file so in order to use it it is sufficient to place it in a directory in the
\TeX\ paths (i.e. the same directory of the \verb'.tex' file which is using
the package).

To start using it place
\begin{latex}
	\usepackage#[/<options>/]#{commenting}
\end{latex}
in your preamble.
Here's the list of available options:
\begin{itemize}
    \item \cmd'draft'/\cmd'final'\doccmd{draft}\doccmd{final} sets on or off
    	the draft-mode. All the comments are omitted when in final-mode.
    	The conditional \cmd'\ifdraft' is made available for explicit detection of the current mode. If the
	    \cmd'draft' option is specified in the \cmd'\documentclass' command it will
	    be detected by the \cmd'commenting' package;
    \item \cmd'nompar'\doccmd{nompar} do not use margin notes and prevents the
    	inclusion of the package \cmd'mparhack';
    \item \cmd'marksonly'\doccmd{marksonly} do not put text in the margin, put
    	only the appropriate mark (see Section~\ref{sec:marks}); this is helpful
    	when the layout requires narrow margins but some hint to quickly
    	find comments is still desirable.
    \item \cmd'nocolor'\doccmd{nocolor} suppress the color changes in
    	commenting commands;
    \item \cmd'nosign'\doccmd{nosign} causes the omission of signatures after
    	every comment;
    \item \cmd'nodate'\doccmd{nodate} suppresses dates in comments and
    	annotations (not in list of comments);
    \item \cmd'noding'\doccmd{noding} do not use dingbats. If this option is
    	specified the package \cmd'bbding' will not be included and replacement
    	symbols are defined;
    \item \cmd'bfstyle', \cmd'sfstyle', \cmd'itstyle', \cmd'scstyle'
    	\doccmd{*style} installs the specified style as default style for comments;
    \item \cmd'pdfcomment'\doccmd{pdfcomment} enables the use of that powerful
    	package: the \cmd'commenting' commands will, in addition to their output,
    	produce a pdf comment (rendered only by an appropriate viewer). Please note
    	that the pdfcomments will be produced in both draft and final mode and
    	regardless of the \cmd'\onlyauthors' directive. To suppress them just
    	remove this option.
\end{itemize}

\subsection{Authors}

You can then specify in the preamble as many authors you need with the commands
\doccmd{\\declareauthor}
\begin{latex}
	\declareauthor{/<nick>/}{/<fullname>/}{/<color>/}
\end{latex}
where \cmd'/<nick>/' is the mnemonic name used to identify the author in the
\LaTeX\ code and must not contain exclamation marks, \cmd'/<fullname>/' is the
name used as a signature, \cmd'/<color>/' is the color used for the text
of the comments of the author.

There exists two special authors:
\begin{itemize}
    \item \cmd'default'\doccmd[(author)]{default} is the author used when no
    	other is specified
    \item \cmd'final'\doccmd[(author)]{final} is the only author whose comments
    	will be printed both in draft and final mode.
\end{itemize}
The default user can be set to be a declared user with
\doccmd{\\setdefaultauthor}
\begin{latex}
	\setdefaultauthor{/<author>/}
\end{latex}
and then it can be reset to the default with
\cmd'\resetdefaultauthor'\doccmd{\\resetdefaultauthor}.

The command\doccmd{\\authorcommand}
\begin{latex}
	\authorcommand{/<author>/}{/<type>/}
\end{latex}
declares a command \cmd'\/<author>/' (and its star variant) that is equivalent
to \mbox{\cmd'\\/<type>/[/<author>/]'}; the argument \cmd'/<type>/' can be equal
to \cmd'comment', \cmd'annot', \linebreak\cmd'changed' or \cmd'draftnote';
for example
\begin{latexex}
	\declareauthor{myself}{Joe}{blue}
	\authorcommand{myself}{comment}
\end{latexex}
will make available the commands \cmd'\myself' and \cmd'\myself*' which can be
used exactly like \cmd'\comment[myself]' and \cmd'\comment*[myself]'.

In the examples we will assume the lines
\begin{latex}
\declareauthor{fz}{Frank}{blue}
\declareauthor{jlp}{Jean-Luc}{green!30!black}
\end{latex}
in the preamble.

With the command\doccmd{\\setauthorstyle}
\begin{latex}
	\setauthorstyle{/<author>/}{/<stylecmds>/}
\end{latex}
different styles can be installed for each declared author; \cmd'/<stylecmds>/'
should contain formatting command such as \cmd'\itshape\sffamily'.
It is also possible to change the base style of every comment with the command
\doccmd{\\setcommentstyle}
\begin{latex}
	\setcommentstyle{/<stylecmds>/}
\end{latex}
keep in mind however that this is overwritten by \cmd'\setauthorstyle' for the
specified author.

You can always refer to an author's color with the named color \cmd'/<nick>/col'
as in\doccmd[(color)]{/<nick>/col}
\begin{latex}
	\color{fzcol}
\end{latex}

The command\doccmd{\\onlyauthors}
\begin{latex}
	\onlyauthors{/<author_1>/,/.../,/<author_n>/}
\end{latex}
enables only the specified authors and mutes all the comments of the other
authors (except \cmd'final').

\section{Usage}

The general syntax for commenting commands is the following:
\begin{latex}
	\command#[/<author>/]#@(/<date>/)@{/<message>/}
\end{latex}
where both \cmd'#[/<author>/]#' and \cmd'@(/<date>/)@' are optional parameters.
If the author is not specified, \cmd'default' would be used instead.
If the date is omitted, no date information will be printed in the comment;
the format of the \cmd'/<date>/' field is explained in Section~\ref{sec:dates}.

In the following we describe all the commands made available by the package
which use this syntax.

\newcommand{\example}[1]{%
\begin{quote}
	Lorem ipsum dolor sit amet, consectetur adipiscing elit. Vivamus orci lorem,
	porttitor #1 non tempus in, porta id tortor. Donec blandit.
\end{quote}%
}%

\subsection{Comments}\doccmd{\\comment}\doccmd{\\comment*}
\begin{latex}
\comment#[/<author>/]#@(/<date>/)@{/<message>/}
\comment*#[/<author>/]#@(/<date>/)@{/<message>/}
\end{latex}
Puts a standard comment in the text.

\textbf{Example:}
\begin{latexex}
\comment[fz](11/04){Jazz is not dead. It just smells funny}
\end{latexex}
\example{\comment[fz](11/04){Jazz is not dead. It just smells funny}}
The star variant \cmd'\comment*' puts every detail about the comment in the text
and does not produce a margin note.

\textbf{Example:}
\begin{latexex}
\comment*[fz](11/04){Jazz is not dead. It just smells funny}
\end{latexex}
\example{\comment*[fz](11/04){Jazz is not dead. It just smells funny}}


Here an example with multiple authors and some omitted parameters. Note the grey
comment by the default author:
\begin{latexex}
  Lorem ipsum dolor sit amet,@\comment{Who am I?}@
  consectetur adipiscing elit. Vivamus orci lorem,
  porttitor non tempus in, porta id tortor. Donec
  blandit @\comment[fz](11/04){Jazz is not dead.
  It just smells funny}@, arcu eget tincidunt dictum,
  @\comment[jlp]{Don't be rude Frank}@ ipsum est
  consequat eros, at iaculis magna erat ac lectus.
\end{latexex}
\begin{quote}
	Lorem ipsum dolor sit amet,\comment{Who am I?} consectetur adipiscing elit.
	Vivamus orci lorem, porttitor non tempus in, porta id tortor. Donec blandit%
	\comment[fz](11/04){Jazz is not dead. It just smells funny},
	arcu eget tincidunt dictum,%
	\comment[jlp]{Don't be rude Frank}%
	ipsum est consequat eros, at iaculis magna erat ac
	lectus.
\end{quote}

\subsection{Annotations}\doccmd{\\annot}
\begin{latex}
\annot#[/<author>/]#@(/<date>/)@{/<message>/}
\end{latex}
Puts a margin note comment on the side of the text. This type of comment is most
useful when one wants to put comments without changing the spacings and the
typesetting of the actual content.

If the default margin spacings are not big enough for your needs you can change
them with\doccmd[(length)]{marginparwidth}
\begin{latex}
	\setlength{\marginparwidth}{/<length>/}
\end{latex}

\textbf{Example:}
\begin{latexex}
\annot[fz](11/04){Jazz is not dead. It just smells funny}
\end{latexex}
\example{\annot[fz](11/04){Jazz is not dead. It just smells funny}}

\subsection{Changes}\doccmd{\\changed}
The command \cmd'\changed' has a slightly different syntax:
\docwarning{Changed:}{optional message comes before content from v1.3}
\begin{latex}
\changed#[/<author>/]#@(/<date>/)@#[/<message>/]#{/<content>/}
\end{latex}
It highlights the portion of text which has been changed by the corresponding
author: the change is typeset with the color of the author and a margin note specifies the
author and date of the change and the optional message.
When the \cmd'nompar' option is loaded, the optional message gets printed inline
as in \cmd'\comment'.

The star variant omits the margin note: this is necessary when using it in
boxes or in math mode.
\doccmd{\\changed*}
\begin{latex}
\changed*#[/<author>/]#@(/<date>/)@{/<content>/}
\end{latex}

\textbf{Example:}
\begin{latexex}
... sit \changed[jlp](11/04)[more appropriate]{amen},
consectetur... Donec \changed[fz](11/04){non} blandit.
\end{latexex}
\begin{quote}
	Lorem ipsum dolor sit
	\changed[jlp](11/04)[more appropriate]{amen},
	consectetur adipiscing elit. Vivamus orci lorem, porttitor non tempus in, porta id tortor.
	Donec \changed[fz](11/04){non} blandit.
\end{quote}
\begin{latexex}
$\sum_j^n\changed*[fz](11/04){\ensuremath{j_{\alpha(z^2)}}}$
\end{latexex}
\begin{quote}
	Lorem ipsum dolor sit $\sum_j^n\changed*[fz](11/04){\ensuremath{j_{\alpha(z^2)}}}$, consectetur adipiscing
	elit.
\end{quote}

\subsection{Remarks}\doccmd{\\draftnote}
\begin{latex}
\draftnote#[/<author>/]#{/<title>/}{/<message>/}
\end{latex}
Puts an indented block of text which is kept separated from the text.

\textbf{Example:}
\begin{latexex}
\draftnote[fz]{Note on Jazz}
	{Jazz is not dead. It just smells funny}
\end{latexex}
\example{\draftnote[fz]{Note on Jazz}{Jazz is not dead. It just smells funny}}

\section{Lists of comments and legend}

You can print all the comments in the order they are inserted with
\doccmd{\\listcomments}
\docwarning{Todo:}{change appearance (toc style) and collection, include page
numbers}
\begin{latex}
	\listcomments
\end{latex}
which in this document would produce
\listcomments

The command\doccmd{\\authorlegend}
\begin{latex}
	\authorlegend
\end{latex}
prints a list of the authors with the corresponding nicks and colors for
reference:
\authorlegend

Both commands are not typeset if not in draft mode.
Their star variants \cmd'\listcomments*' and \cmd'\authorlegend*'
\doccmd{\\listcomments*}\doccmd{\\authorlegend*} print their
output in both modes.

\section{Dates}\label{sec:dates}
Almost all of the commands presented above have a field for specifying the date
of the comment. The content in this field can have \emph{any} form.
There are two commands that will require to specify dates in a certain format:
\cmd'\NoCommentsBefore' and \cmd'\NoCommentsAfter'; if you do not use these
macros you do not need to follow any particular format.

In the case you want to use one of the commands used in this section, the date
field will be interpreted as a date-like value, the format of which is very
generic: by default it is a list of numbers separated by \cmd[style=noslash]'/'
as for example in \cmd[style=noslash]'01' or \cmd[style=noslash]'11/7' or
\cmd[style=noslash]'25/12/2000' or even \cmd[style=noslash]'1/2/0323/8/11/9'.
This sequence will be printed as it is in the document in the appropriate place
with the corresponding author style (if not suppressed by \cmd'nodate'). These
dates will be interpreted as following: the first number in the sequence is the
less significant, increasing significance until the end of the list; a format
following this convention is for example \texttt{dd/mm/aa}; it is not required
to write every number with two digits. In general, the interpretation of a
generic sequence \cmd[style=noslash]'!n_0!/!n_1!/.../!n_k!' is the number
$n_0 + 10^{\max(2,|n_0|)}\times n_1 +
	\ldots +10^{\max(2k,\sum_{i}^{k-1}|n_{i}|)}\times n_k$.
This enables the use of the commands \cmd'\NoCommentsBefore{/<date>/}' and
\cmd'\NoCommentsAfter{/<date>/}'\doccmd{\\NoCommentsBefore}\doccmd{\\NoCommentsAfter}:
they suppress any comment before/after the specified date.

By default not dated
comments are included even if before/after dates are set; with the macro
\cmd'\ExcludeNonDatedComments'\doccmd{\\ExcludeNonDated...} they
will be suppressed if a before/after date is set;
\cmd'\IncludeNonDatedComments'\doccmd{\\IncludeNonDated...} reverts to the
default behaviour.

For example, the following code
\begin{latexex}
-\comment*[fz](a.XXX-c){alien date, not parsed}\\
@\NoCommentsBefore{12/04}@
@\NoCommentsAfter{20/04}@
-\comment*[fz](10/04){early comment}\\
-\comment*[fz](29/4){suppressed because after 20/04}\\
-\comment*[fz]{included because not dated}\\
@\ExcludeNonDatedComments@
-\comment*[fz]{suppressed because not dated}\\
-\comment*[fz](15/04){included because after 12/04}\\
@\NoCommentsBefore{}@ % suppresses the before bound
-\comment*[fz](19/04){included because no before bound is set}\\
-\comment*[fz](29/04){suppressed because after 20/04}\\
@\NoCommentsAfter{}@ % suppresses the after bound
-\comment*[fz](bla){another non parsed date, no bound set}
\end{latexex}
outputs
\begin{quote}
-\comment*[fz](a.XXX-c){alien date, not parsed}\\
\NoCommentsBefore{12/04}
\NoCommentsAfter{20/04}
-\comment*[fz](10/04){early comment}\\
-\comment*[fz](29/4){suppressed because after 20/04}\\
-\comment*[fz]{included because not dated}\\
\ExcludeNonDatedComments
-\comment*[fz]{suppressed because not dated}\\
-\comment*[fz](15/04){included because after 12/04}\\
\NoCommentsBefore{} % suppresses the before bound
-\comment*[fz](19/04){included because no before bound is set}\\
-\comment*[fz](29/04){suppressed because after 20/04}\\
\NoCommentsAfter{} % suppresses the after bound
-\comment*[fz](bla){another non parsed date, no bound set}
\end{quote}

The command \cmd'\SetCommentDateDelim{/<delim>/}'\doccmd{\\SetCommentDateDelim}
can be invoked to change the delimiter of a date:
\begin{latexex}
\SetCommentDateDelim{.}
\NoCommentsBefore{08.2008}
\comment*[jlp](1.2008){suppressed}
\comment*[jlp](10.2008){included}
\end{latexex}

For obvious reasons, it is recommended to use a uniform date format convention
within a document; however it is possible to change delimiter and range in any
point of the document: the changes will have effect from that point through the
rest of the document.

\section{Other commands}

There are a couple of more commands which do not follow the `author' system.
They are used to typeset generic remarks, notes that one want to put
after a paragraph but which are meant to be read only by other authors.

The simple \cmd'\ondraft{/<text>/}'\doccmd{\\ondraft} command prints its
argument only when in draft mode. Analogously the
\cmd'\onfinal{/<text>/}'\doccmd{\\onfinal} command prints its argument only when
in final mode.

In order to insert artificial spacing between comments that does not get produced
when not in draft mode, the package defines the commands
\cmd'\draftnl'\doccmd{\\draftnl} for newline and
\cmd'\draftpar'\doccmd{\\draftpar} for paragraph.
Compare the two outputs:\\
\begin{minipage}{.5\linewidth}
\begin{latexex}
\comment*[fz]{Can I tell you something Jean-Luc?}
\comment*[jlp]{Yes, go ahead}
\comment*[fz]{You stink}
\end{latexex}
\end{minipage}\hfill
\begin{minipage}{.4\linewidth}
\small
\comment*[fz]{Can I tell you something Jean-Luc?}
\comment*[jlp]{Yes, go ahead}
\comment*[fz]{You stink}
\end{minipage}

\begin{minipage}{.5\linewidth}
\begin{latexex}
\comment*[fz]{Can I tell you something Jean-Luc?}
\draftnl
\comment*[jlp]{Yes, go ahead}
\draftnl
\comment*[fz]{You stink}
\end{latexex}
\end{minipage}\hfill
\begin{minipage}{.4\linewidth}
\small
\comment*[fz]{Can I tell you something Jean-Luc?}\draftnl
\comment*[jlp]{Yes, go ahead}\draftnl
\comment*[fz]{You stink}
\end{minipage}

The \cmd'Note' environment\doccmd[(environment)]{note}
\docwarning{Changed:}{note renamed to Note to avoid clashes with common packages
from v1.3}
\begin{latex}
	\begin{Note}#[/<signature>/]#{/<title>/}
		/<message>/
	\end{Note}
\end{latex}
is similar to \cmd'\draftnote' but has no author and is typeset in both draft and
final modes.

%\pagebreak[4]
\textbf{Example:}
\begin{latexex}
\begin{Note}[The Authors]{Note on Jazz}
Jazz is not dead. It just smells funky.
\end{Note}
\end{latexex}
\example{\begin{Note}[The Authors]{Note on Jazz}
Jazz is not dead. It just smells funky.
\end{Note}}

The \cmd'todo' command
\begin{latex}
	\todo#[/<annotation>/]#{/<message>/}
\end{latex}
puts a box in the middle of the text as a placeholder for
something that will be written in the future. This command will not print
anything when in final mode.

\textbf{Example:}
\begin{latexex}
\todo[For Frank]{The best jazz smell in the world.\\Frank, can you do it?}
\end{latexex}
\example{\todo[For Frank]%
	{The best jazz smell in the world.\\Frank, can you do it?}}

%%% UNDOCUMENTED
% \example{\begin{TodoList}{For Frank}%
% 	\item The best jazz smell in the world.\item Frank, can you do
% 	it?\end{TodoList}}

The color of \cmd'todo' boxes is named `\cmd'todocol'' and can be changed at any
time with \cmd'\colorlet{todocol}{/<color>/}'.

\section{Symbols}\label{sec:marks}

Several commands can be overridden (or used) to control the typesetting of
symbols used by the package:
%\begin{table}[htb]
\begin{center}
	\begin{tabular}{rcc}
	\textbf{Command} & \textbf{Symbol} & \textbf{with \texttt{noding}}\\
	\cmd'\CommentMarkR' & \CommentMarkR & \CommentMarkR \\
	\cmd'\CommentMarkL'  & \CommentMarkL & \CommentMarkL \\
	\cmd'\OpenCommentBraket' & \OpenCommentBraket & \OpenCommentBraket \\
	\cmd'\ClosedCommentBraket' & \ClosedCommentBraket & \ClosedCommentBraket \\
	\cmd'\NoteMark' & \NoteMark & $\star$ \\
	\cmd'\ChangedMarkR' & \ChangedMarkR & \textasteriskcentered \\
	\cmd'\ChangedMarkL' & \ChangedMarkL & \textasteriskcentered\\
	\cmd'\TodoMarkR' & \TodoMarkR & \textdagger\\
	\cmd'\TodoMarkL' & \TodoMarkL & \textdagger\\
	\cmd'\AuthorMark' & \AuthorMark & \textbullet
	\end{tabular}
\end{center}
%\caption{Symbols defined and used by the package}
%\label{tab:symbols}
%\end{table}
 There are few named colors that can be redefined:
\cmd'draftcol'\doccmd[(color)]{draftcol} (\cmd'gray' by default) and
\cmd'finalcol'\doccmd[(color)]{finalcol} (\cmd'black' by default) are the colors
of the corresponding special authors;
\cmd'todocol'\doccmd[(color)]{todocol} (\cmd'violet' by default) is the color of
\cmd'\todo' boxes.

\end{document}
